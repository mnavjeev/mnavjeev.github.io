%!TEX root = /Users/manunavjeevan/Desktop/Research/Moment Inequality Methods/Annotated Literature Review/inequalityLitReview.tex

\newpage
\section{A Geometric Approach to Inference in Set-Identified Entry Games; \textit{\small Christian Bontemps, Rohit Kumar (JoE, 2020)}}\label{sec:BK-2020}

\citet{BK-2020} is set to appear in Journal of Econometrics in 2020. They consider inference procedures for entry games with complete information. Complete the model with the unknown selection mechanism and characterize geometrically the set of predicted choice probabilities. 

\subsection{Introduction}

Paper provides an estimation procedure for empirical models of entry and market structure. Entry games are popular in the empirical Industrial Organization literatures because they allow researchers to study the nature of firms' profits and the nature of competition between firms from data that are generally easy to collect. Popularized by the seminal works of Bresnahan and Reiss (1991a). 

Econometrics analysis of entry games is complicated by the presence of multiple equilibria, a problem that affects the standard estimation strategy.  Without additional assumptions, the model is incomplete. 

This paper completes the model with the selection mechanism $\eta(\cdot)$ and characterizes the set of predicted choice probabilities generated by the variation of $\eta(\cdot)$ in the space of admissible selection mechanisms. First contributions is to characterize more deeply the geometric structure of this set. 

Set ends up being a convex polytope with many facets (because of focus on pure strategy equilibrium). This paper derived a closed form expression for the support function of this polytope, the extreme points (or \emph{vertices}) of which can also be calculated as a function of the primitives of the model. Vertices are characterized by an order of outcome selection in the regions of multiple equilibria. Each vertex is also geometrically defined by the intersection of some supporting hyperplanes. Able to define the cone of outer normal vectors of these hyperplanes, and, thereby, the inequalities that are binding in this point.

Testing whether a parameter belongs to the identified set is equivalent to testing whether the true choice probability vector belongs to this convex set. However, when the number of players increases, the number of facets of the polytope increases exponentially, and, therefore, the smallest number of inequalities necessary to have a sharp characterization of the the identified set - from 16 in a game with 3 players to more than 1 million in a game with 6 players.

\subsection{Entry Game with \emph{N} players}

Formalize the entry game considered with $N$ firms. First consider a model without explanatory variables.

\subsubsection{Setup and Notations}

\paragraph{Model}
Let $N$ denote the number of firms that can enter any market. Following Berry (1992), introduce a model of market structure where the profit function $\pi_{im}$ of firm $i$ in a market $m$ is assumed to be independent of the identity of the firm's competitors. All firms decide simultaneously whether to enter the market (action \(a_{im} = 1\), doing so if their profit is positive. If $\pi_{im} = 0$, firms do not enter the market (take action \(a_{im}= 0\)). The profit function is assumed, without loss of generality, to be linear in the explanatory variables\footnote{[FOOTNOTE FROM TEXT] Any separable parametric form $\pi_{im} = f_i(\sum_{j\neq i} a_{jm}l\alpha) + \eps_{im}$ can be considered as long as the function $f_i(\cdot;\theta)$ is strictly decreasing in its first argument.}

Following the literature, assume that $\alpha_i < 0$, i.e, the presence of more competitors decreases a firm's profit. Unobserved components $\eps_{im}, i = 1,\dots, N$ are drawn from a known distribution (up to some parameter vector $\gamma$). Econometrician does not observe their values but firms do.

For identification, we first need a scale normalization, and thus, assume that the variance of each shock $\eps_{im}$ is equal to 1. Denote the distribution of $\eps_m = (\eps_{1m},\dots, \eps_{Nm})^T$ and assume that the distribution is continuous with full support. Use notation $\theta$ for all the parameters in the model, and omit the subscript $m$ for notational convenience. Assume $\theta \in \Theta \subseteq \SR^l$. 

\paragraph{Multiplicity of Pure Strategy eqm.}

For a given market, an outcome $y$ is the vector of actions (in $\{0,1\}^N$) taken by the firms. There are $2^N$ possible outcomes. Denote by $\calY$ this set of possible outcomes. $\calY_K$ denotes the subset of outcomes with $K$ active firms in equilibrium, i.e any firms playing action 1. There is 1 outcome with 0 active firms, $N$ outcomes  with 1 active firm, $d_k = \binom{N}{K}$ with $K$ active firms, etc. 

Globally, order the outcomes in $\calY$ first by their number of active firms, then according to the predefined order within each $\calY_K$:
\[\calY = \left\{\underbrace{y_1^{(0)}\vphantom{,\dots,y_{d_K}^{(N)}}}_{\calY_0}, \underbrace{y_1^{(1)}, \dots, y_{d_1}^{(1)}}_{\calY_1}, \dots, \underbrace{y_1^{(K)},\dots,y_{d_K}^{(K)}}_{\calY_K},\dots, \underbrace{\vphantom{,\dots,y_{d_K}^{(N)}}y_1^{(N)}}_{\calY_N} \right\}\]

Well known that the model has multiple equilibria, regions of realizations of $\eps$ in which one cannot uniquely predict each firms' action. Consequently, no one-to-one mapping between the collection of possible outcomes and the regions of $\eps$ given any parameter value $\theta$. Consequently no one-to-one mapping between the collection of possible outcomes and the regions of $\eps$ given any parameter value $\theta$. 

Missing from the model is the selection of a given equilibrium in the regions of multiple equilibria. Define this selection mechanism $\eta(\cdot)$ as in Definition \ref{def:GH-2} of \citet{GH-2011}.

\begin{definition}[Equilibrium Selection Mechanism]
	\label{def:BK-1}
	An equilibrium selection mechanism is a conditional probability $\eta(\cdot | \eps; \theta)$ such that the selected value of the outcome variable is actually an equilibrium predicted by the game. That is $\supp(\eta(\cdot | \eps;\theta)) \subseteq G(\eps |\theta) $
\end{definition}

Denote by $\calE$ the set of selection mechanisms and by \(P(\theta,\eta)\) the predicted choice probability vector when the parameter of the model is is \(\theta\) and selection mechanism is $\eta(\cdot)$. Partition this vector according to the partition of $\calY$ as 
\begin{equation}
	\label{eq:BK-2}
	P(\theta,\eta) = \left(\underbrace{\vphantom{,\dots P_{d_k}^{(K)}(\theta,\eta)}P_1^{(0)}(\theta,\eta)}_{P^{(0)}(\theta,\eta)},\dots, \underbrace{P_1^{(K)}(\theta,\eta),\dots, P_{d_K}^{(K)}(\theta,\eta)}_{P^{(K)}(\theta,\eta)},\dots\underbrace{\vphantom{,\dots P_{d_k}^{(K)}}P_1^{(N)}(\theta,\eta)}_{P^{(N)}(\theta,\eta)}\right)^T
\end{equation}

One solution to the multiple equilibria problem consists of making assumption on this selection mechanism like in Reiss (1996) or Cleeren (2010). The vector of predicted choice probabilities is a point in $[0,1]^{2^N}$ and standard inference techniques may be used. Of course, any restrictions are ad hoc and may lead to misspecification. 

Another solution, following the literature on set-identification, consists of characterizing all the possible choice probabilities predicted by the model. The vector of predicted choice probabilities, instead of being an unrestricted point, belongs to a convex set that is characterized. Different sets of values $(\theta, \eta)$ may generate the same point $P(\theta,\eta)$. 




