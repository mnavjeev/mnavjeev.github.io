%!TEX root = /Users/manunavjeevan/Desktop/Research/Identification/Annotated Literature Review/identificationLitReview.tex

\section{An Almost Ideal Demand System}

Here I read the 1980 Paper, \emph{An Almost Ideal Demand System} By Angus Deaton and John Muelbauer. The paper goes over how to identify demand systems in large markets. The hope is to adapt this somehow to two sided markets under competition.

\subsection{Introduction}
\begin{itemize}
	\item Richard Stone (1954) first estimated a system of demand equations derived explicitly from consumer theory
	\item Paper proposes and estimates a new model which is of comparable generality to the Rotterdam and translog models but has advantages over both
	\item Model (AIDS) gives an arbitrarty first order-approximation to any demand system, satisfies the axioms of choice exactly, aggregates perfectly over consumers without invoking parallel linear Engle curves, has a functional form which is consistent with known household-budget data, simple to estimate, largely avoids need for non-linear estimation.
	\item Model is estimated on postwar British data.
\end{itemize}

\subsection{Specification of the AIDS}

Generally, starting point has been the specification of a function general enough to act as a second-order approximation to any arbitrart direct or indirect utility function. It is possible to use a first order approximation to the demand functions themselves as in the Rotterdam model. 

AIDS approach follows from these approaches but builds froma specific class of preferences by which the theorems of Muellbauer permit exact aggregation over consumers. Preferences are known as PIGLOG preferences and are represented via expenditure function\footnote{minimum expenditure necessary to obtain a specific utility level at given prices}. Denote the expenditure function $c(u,p)$ for utility $u$ and price vector $p$ and define the PIGLOG class by
\begin{equation}
	\label{eq:AIDS-1}
	\log c(u,p) = (1-u)\log a(p) + u \log b(p); \hbox{ }\text{ for }\hbox{ }u\in[0,1]
\end{equation}
The positive constant returns to scale functions $a(p)$ and $b(p)$ can be regarded as the costs of substinence and bliss, respectively. Next take specific functional forms for $\log a(p)$ and $\log b(p)$. To ensure a flexible functional form for the cost function, the specific funcitonal form for the above functions must posess enough parameters so that at any single point the cost function derivatives up to a second order w.r.t $u$ and $p$ can be set equal to those of an arbitrary cost function\footnote{Approximation is at a point, basically means that the model specification gives the ``specifier'' full control over the first and second derviatives \emph{at one specific point}}. Take
\begin{align}
    \label{eq:AIDS-2}
    \log a(p) &= \alpha_0 + \sum_k \alpha_k \log p_k + \frac{1}{2}\sum_k \sum_j \gamma^*_{kj} \log p_k \log p_j \\
    \label{eq:AIDS-3}
    \log b(p) &= \log a(p) + \beta_0 \prod_k p_k^{\beta_k}
 \end{align}
 With these specifications, the AIDS cost function is written 
 \begin{equation}
 	\label{eq:AIDS-4}
 	\log c(u,p) = \alpha_0 + \sum_k \alpha_k \log p_k + \frac{1}{2}\sum_k \sum_j \gamma^*_{kj} \log p_k \log p_j + u\beta_0 \prod_k p_k^{\beta_k}
 \end{equation}
 where $\alpha_i, \beta_i, \gamma^*_{ij}$ are the parameters of the model. Clearly, $c(u,p)$ is CRS in $p$ provided that $\sum_i \alpha_i = 1, \sum_j \gamma^*_{kj} = \sum_k \gamma^*{kj} = \sum_j \beta_j = 0$. Also straightforward to check that (\ref{eq:AIDS-4}) has enough paramaters to be a flexible functional form remembering that utility is orginal so we can always choose a normalization such that, at a point $\pderiv[2]{\log c}{u} = 0$. 

 Demand equations can be derived from the cost functions in equation (\ref{eq:AIDS-4}). It is a fundamental property of the cost function that its price derivatives are the quantities demanded $\pderiv{c(u,p)}{p_i} = q_i$. Multiplying both sides by $p_i/c(u,p)$ delivers 
 \begin{equation}
 	\label{eq:AIDS-5}
 	\pderiv{\log c(u,p)}{p_i} = \frac{p_i q_i}{c(u,p)} = w_i
 \end{equation}
 where $w_i$ is the budget share of good $i$. So log differentiation of (\ref{eq:AIDS-4}) gives budget shares as a function of prices and utility 
 \begin{equation}
 	\label{eq:AIDS-6}
 	w_i = \alpha_i + \sum_j \gamma_{ij} \log p_j + u \beta_0\beta_i \prod_k p_k^{\beta_k}
 \end{equation}
 where $\gamma_{ij} = \frac{1}{2}(\gamma^*_{ij} + \gamma^*_{ji})$\footnote{If the Slutsky matix is symmetric $\gamma_{ji}^* = \gamma_{ij}^* = \gamma_{ij}= \gamma_{ji}$ anyways.}. For a utility mazimizing consumer, total expenditure $x$ is equal to $c(u,p)$ and this equality can be inverted to five $u$ as a function of $p$ and $x$, the indirect utility function. Do this for (\ref{eq:AIDS-4}) and substitute into (\ref{eq:AIDS-6}) to obtain budget shares as a function of prices (\(p\)) and total expenditure (\(x\)); these are the AIDS demand functions in budget share form. 
 \begin{equation}
 	\label{eq:AIDS-8}
 	w_i = \alpha_i + \sum_j \gamma_{ij}\log p_j + \beta_i \log (c/P)
 \end{equation}
 where $P$ is a price index defined by 
 \begin{equation}
 	\label{eq:AIDS-9}
 	\log P = \alpha_0 + \sum_k \alpha_k \log p_k + \frac{1}{2} \sum_{j,k} \gamma_{kj} \log p_k \log p_j
 \end{equation}
 The restrictions on the parameters of (\ref{eq:AIDS-4}) plus the definition of $\gamma_{ij}$ imply restrictions on the parameters of the AIDS equation (\ref{eq:AIDS-8}). Specifically, they require
 \begin{align}
     \label{eq:AIDS-10}
     \sum_i \alpha_i = 1,\hbox{ }  \sum_i \gamma_{ij} &= 0 ,\hbox{ } \sum_i \beta_i = 0 \\
     \label{eq:AIDS-11}
     \sum_j \gamma_{ij} &= 0 \\
     \label{eq:AIDS-12}
     \gamma_{ij} &= \gamma_{ji}
 \end{align}
 provided (\ref{eq:AIDS-10}), (\ref{eq:AIDS-11}), (\ref{eq:AIDS-12}) hold, (\ref{eq:AIDS-8}) represents a system of demand functions which add up to total expenditure (\(\sum w_i = 1\)), are homogeneous of degree 0 and which satisfy Sltusky symmetry.

 Given these, the AIDS is simply interpreted: in the absence of changes in relative prices and the real expenditure $(x/P)$, the budget shares are constants. This is a starting point for predictions using the model\footnote{This seems like a testable resttiction of the model}. 

 Changes in relative prices work through the terms $\gamma_{ij}$: each \(\gamma_{ij}\) represents $10^2$ the effect on the $i$-th budget share of a 1 percent increse in the $j$-th price with \((x/P)\) held constant. Changes in real expenditure operate through the $\beta_i$ coefficients; these add to zero and are positive for ``luxuries'' and negative for ``necessities''.

\subsection{Aggregation Over Households}

\emph{Note: This section basically shows that the AIDS demand system derived above for a single household aggregates nicely under some assumptions, and that the aggregate demand curve looks basically like the individual household demand curve. In general, I don't think that microfounding the aggregate demand model in this way is so important, espcially given the strict assumption required to do this, but it does provide some intuition as to underlying assumptions that may be useful.}

Aggregation theory from Muellbauer (1975, 1976) imply that exact aggregation is possible if, for each household $h$, behavior is descrbed by a generalization of (\ref{eq:AIDS-8}) given below\footnote{Interesting to note which parameters depend on $h$ here}.
\begin{equation}
	\label{eq:AIDS-8'}
	w_{ih} = \alpha_i + \sum_j \gamma_{ij} \log p_j + \beta_i \log(x_h/k_h P)
\end{equation}
$k_h$ can be interpreted as a sophisticated measure of households size, which, in principle, could account for age composition, other household characteristics, and economies of household size. This allows for a limited amount of taste variation across households. 

The share of aggregate expenditure on good $i$ in the aggregate budget of all households, denoted $\bar{\omega}_i$ is given by 
\[\sum_h p_i q_{ih} \bigg/\sum_{x_h} = \sum_h x_j w_{ih}\bigg/\sum x_h\]
Applied to (\ref{eq:AIDS-8'}) this yields 
\begin{equation}
	\label{eq:AIDS-8''}
	\bar{w}_i = \alpha_i + \sum_j \gamma_{ij} \log p_j - \beta_i \log P + \beta_i \left\{\sum_h x_j \log (x_h/k_h)\bigg/\sum x_h\right\}
\end{equation}
Define the aggregate index $k$ by
\begin{equation}
	\label{eq:AIDS-13}
	\log(\bar{x}/k) := \sum_h x_h \log(x_h/k_h)\bigg/\sum x_h
\end{equation}
where $\bar{x}$ is the average level of total expenditure $x_h$. So (\ref{eq:AIDS-8''}) becomes 
\begin{equation}
	\label{eq:AIDS-8'''}
	\bar{w}_i = \alpha_i +\sum_j \gamma_{ij} \log p_j + \beta_i \log(\bar{x}/kP)
\end{equation}
Notice this is the same form as in (\ref{eq:AIDS-8'}) and shows that under these assumptions, aggregate budget shares correspond to the decisions of a representative household whose budget is given by $\bar{x}/k$, the ``representative'' budget level.

When estimating the model, it is generally assumed that $k$ is constant or uncorrelated with $\bar{x}$ or $p$ so that no omitted variably bias occurs from omitting $k$ from estimation and redefining $\alpha_i^* = \alpha_i - \beta_i \log k^*$, where $k^*$ is the constant or sample mean value of $k$.

\subsection{Generality of the Model}

Flexible functional form property of the AIDS cost function implies that the demand functions derived from it are first order approximations to any set of demand functions derived from utility maximizing. If mazimizing behavior is not assumed but demands are continuous functions of the budget and prices, then AIDS demand functions (\ref{eq:AIDS-8}) can still be viewerd as a first-order approximation\footnote{I think this is the best interpretation of the AIDS model.}. Because AIDS model still gives full control over first and second derivatives, this is not bad. 

However, there are still a lot of parameters. One obvious restriction is that, for some pairs $(i,j), \gamma_{ij}$ should be zero. For such pairs, the budget share of each is independent of the price of the other if $(x/P)$ is held constant. Can be shown that $\gamma_{ij}$ has approximately the same sign as the compensated cross-price elasticity\footnote{This paper is published in 1980, well before the 1996 Tibshirani LASSO paper. Restricting some of the $\gamma_{ij}$ to be 0 is essentially a sparsity condition. It would be interesting to come up with a way of applying LASSO here. Maybe also take a Bayseian approach. The researcher has some prior on which products are 0, use LASSo to update the prior.}. 

\subsection{Restrictions}

Starting from equations (\ref{eq:AIDS-8}) and (\ref{eq:AIDS-9}) as maintained hypotheses can examine the effects of restrictions (\ref{eq:AIDS-10})-(\ref{eq:AIDS-12}). The conditions (\ref{eq:AIDS-10}) are the adding-up restrictions and, as can be checked from (\ref{eq:AIDS-8}), they ensure that $\sum w_i = 1$.

Homegeneity of the demand functions require restriction (\ref{eq:AIDS-11}) which can be tested equation by equation. Symmetry is also a testable restriction.

\subsection{Estimation}
\label{sec:AIDS-1.D}

In general, estimation can be carried out by substituting (\ref{eq:AIDS-9}) into (\ref{eq:AIDS-8}) to give
\begin{equation}
	\label{eq:AIDS-15}
	w_i = (\alpha_i - \beta_i\alpha_0) + \sum_j \gamma_{ij} \log p_j + \beta_i \left\{\log x - \sum_k \alpha_k \log p_k - \frac{1}{2}\sum_{k,j} \gamma_{kj} \log p_k \log p_j\right\}
\end{equation}
and estimating this non-linear system of equations by maximum likelihood or other methods with and without the restrictions (\ref{eq:AIDS-11}) and (\ref{eq:AIDS-12}). With rnough data, this is not particularly difficult to estimato since the first order conditions for MLE are linear in $\alpha$ and $\gamma$, given $\beta$, and vice versa so that concentration allows iteration on a subset of the parameters. 

Although all the parameters in (\ref{eq:AIDS-15}) are identified given sufficient variation in the independent variables, in many examples the practical identification of $\alpha_0$ may be difficult. The parameter is only identified from the $\alpha_i$'s in (\ref{eq:AIDS-15}) by the presence of these latter inside the term in braces, originally in the formula for $\log P$. However, in situations where individual prices are closely collinear, $\log P$ is unlikely to be very sensitive to its weights so that changes in the intercept term in (\ref{eq:AIDS-15}) due to variations in $\alpha_0$ can be offset in the $\alpha$'s with minimal effect on $\log P$. This can be overcome in practice by assigning a value to $\alpha_0$ a priori. 

In many situations it is possobile to explot the collinearity of prices for a much simplet estimation technique. Note from (\ref{eq:AIDS-8}) that if $P$ is known, the model is linear in it's parameters and so estimation can be done equation by equation through OLS (at least without cross-equation restrictions such as symmetry.)

The adding-up constraints (10) will be automatically satisfied by these estimates. In situations where prices are closely collinear, it may be adequate to approximate $P$ as proportional to some index $P^*$ such as $\log P^* = \sum w_k \log p_k$. In this case, (\ref{eq:AIDS-8}) can be estimated as 
\begin{equation}
	\label{eq:AIDS-16}
	w_i = (\alpha_i - \beta_i \log \phi) + \sum_j \gamma_{ij} \log p_j + \beta_i \log\left(\frac{x}{P^*}\right)
\end{equation}
In this setup, the $\alpha_i$ parameters are only identifies up to a scalar multiple of $\beta_i$\footnote{because we don't know what $\phi$ is presumably}. If we write $\alpha_i^* = \alpha_i - \beta_i \log \phi$, it is easily seen that $\sum \alpha_k^* = 0$ is required for adding up, since $\sum \beta_k = 0$. 

Empirical results show that (\ref{eq:AIDS-16}) is a good approximation for (\ref{eq:AIDS-15}). However, it is still an approximation.

\subsection{An Application to Postwar British Data}

Estimate the model using annual British daa from 1954 to 1974 on eight nondurable groups of consumer expenditure, namely: food, clothing, housing services, fuel, drink + tobacco, transport + communication services, other goods, and other services.

As discussed above, if we assume that the index $k$ in (\ref{eq:AIDS-8'''}) is either constant or that its deviations are independently distributied from those of the average budget $\bar{x}$ and of prices, no biases result from its omission. In particular, allow the intercepts in (\ref{eq:AIDS-8'''}) to absorb the $-\beta_i \log k$ terms. Then proceed by first following the strategy outlined in Section \ref{sec:AIDS-1.D}, setting $\log P^* = \sum w_k \log p_k$ for each year and estimation equation (\ref{eq:AIDS-16}).








