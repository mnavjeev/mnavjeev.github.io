%!TEX root = /Users/manunavjeevan/Documents/GitHub/mnavjeev.github.io/files/Functional Analysis/rudin.tex

\section{Chapter 1: Topological Vector Spaces}

\subsection{Introduction and Normed Spaces}

Many problems require analysis of classes of functions. Most interesting classes are vector spaces with normed. 

A vector space $X$ is said to be a \emph{normed space} if, for every $x\in X$ there is a nonnegative real number $\|x\|$, called the \emph{norm} of $x$, such that
\begin{enumerate}
	\item \(\|x+y\| \leq \|x\| + \|y\|\), for all $x,y \in X$
	\item \(\|\alpha x\| = |\alpha|\|x\|\), for $x \in X$ and $\alpha \in \SR$
	\item \(\|x\| > 0\) if $x\neq 0$
\end{enumerate}
The \emph{norm} is the \emph{function} that maps $x$ to $\|x\|$. Every normed space can be regarded as a metric space, in which the distance is determined by  $d(x,y) = \|x-y\|$. The relevant properties of $d(x,y)$ between $x$ and $y$ is $\|x-y\|$. The relevant properties of $d$ are 
\begin{enumerate}
	\item \(0 \leq d(x,y) < \infty\) for all $x$ and $y$
	\item \(d(x,y) = 0\) if and only if $x = y$
	\item \(d(x,y) = d(y,x)\) for all $x,y$
	\item \(d(x,z) \leq d(x,y) + d(y,z)\) for all \(x,y,z\) 
\end{enumerate}
In any metric space, the \emph{open ball} with center at \(x\) and radius \(r\) is the set 
\[B_r(x) = \{y: d(x,y) < r\}\]
In particular, if $X$ is a normed space, the sets 
\[B_1(0) = \{x: \|x\| < 1\}\hbox{ }\text{ and }\hbox{ }\bar{B}_1(0) = \{x: \|x\|\leq 1\}\]
are the open unit ball and closed unit ball of $X$, respectively. We can form a \emph{topology} on $X$ by declaring a set open if and only if it is a (possibly empty) union of open balls. 
\begin{comment*}
	It is easy to verify that the vector spece operations (addition and scalar multiplication) are continuous in this topology. A continuous function maps open sets to open sets. The set of open balls forms a basis of the topology. To verify continuity, we only need to verify this property for elements of the basis. 
	\begin{enumerate}
	 	\item For $\alpha \in \SR$, $\alpha(B_r(x)) = B_{\alpha r}(\alpha x)$
	 	\item Notate $x + y$ as a function $+: X\times X \to X$. For $x,y$ we have $+(B_r(x) \times B_{r'}(y)) = B_{r+r'}(x+y)$.
	 \end{enumerate} 
\end{comment*}
A \emph{Banach space} is a normed space which is \emph{complete}, which means that all Cauchy Sequences converge. Most of the best-known function spaces are Banach spaces. For example, 
\begin{itemize}
	\item The set of all continuous functions on compact spaces
	\item $L^p$ spaces that occur in integration theorem
	\item Hilbert spaces
\end{itemize}
\subsection{Vector Spaces}
The letters $\SR$ and $\SC$ will denote the real and complex numbers, respectively. For the moment, let $\Phi$ stand for either $\SR$ and $\SC$. A scalar is a member of the scalar field $\Phi$. A vector space of $\Phi$ is a set $X$ whose elements are called vectors and in which two operations, \emph{addition} and \emph{scalar multiplication} are defined with the following algebraic properties 
\begin{enumerate}
	\item To every pair of vectors $x$ and $y$ corresponds a vector $x+y$ in such a way that 
	\[x+y=y+x\hbox{ }\text{ and }\hbox{ } x + (y + z) = (x+y) + z\]
	\item To every pair $(\alpha, x) \in \Phi \times X$, there corresponds a vector $\alpha x$ in such a way that 
	\[1x = x\hbox{ }\text{ and }\hbox{ }\alpha(\beta x) = (\alpha \beta)x\]
	and so that the two dsitributive laws hold
	\[\alpha(x+y) = \alpha x + \alpha y\andbox(\alpha + \beta)x = \alpha x + \beta y\]
	\(\and\)
\end{enumerate}
The symbol $0$ will also be used for the zero element of the scalar field. A real vector space is one for which $\Phi = \SR$ and a complex vector space is one for which $\Phi = \SC$. 

If $X$ is a vector space, $A,B\subset X$, $x\in X$ and $\lambda \in \Phi$, the following notation is used
\begin{align*}
    x + A &= \{x + a : a \in A\} \\
    x - A &= \{x + (-1a): a \in A\}\\
    A + B &= \{a + b: a\in A, b\in B\}\\
    \lambda A &= \{\lambda a: a\in A\}
\end{align*}
Not from these conventions that it need not be that $2A = A + A$.\footnote{For example, if $A = \{1,3\}$, then $2A = \{2, 6\} \neq \{2, 4, 6\} = A + A$. We should, however, always have that $2A \subset A + A$.} 

A set $Y \subset X$ is called a subspace of $X$ if $Y$ is itself a vector space. Importantly this means closed under scalar multiplication and addition. One can check that this happens if and only if $0 \in Y$ and 
\[\alpha Y + \beta Y \subset Y\]
for all scalars $\alpha, \beta \in \Phi$.

A set $C \subset X$ is said to be {\it convex} if 
\[tC + (1-t)C \subseteq C, (0 \leq t \leq 1)\]
A set $B \subset X$ is said to be balanced if $\alpha B \subset B$ for every $\alpha \in \Phi$ with $|\alpha|\leq 1$.\footnote{This would be like decreasing returns to scale on a production possibilities set.} A vector space has {\it dimension n} if it has a basis \(\{u_1, \dots, u_n\}\).\footnote{Every $x\in X$ can be written as a linear combination of basis elements}.

\subsection{Topological Spaces}

A {\it topological space} is a set $S$ in which a collection $\tau$ of subsets (call {\it open sets}) has been specified with the following properties 
\begin{enumerate}
	\item $S, \emptyset$ are open
	\item The intersection of any two open sets is open
	\item THe union of every collection of open sets is open 
\end{enumerate}
A set $E \subset S$ is {\it closed} if and only if it's complement is open. The {\it closure}, $\bar{E}$, of $E$ is the intersection of all closed sets that contain $\bar{E}$\footnote{An equivalent definition of this is the smallest closed set that contains $E$. The intersection of an arbitrary collection of closed sets is closed, this is clear from taking the ``complement'' of the fact that an arbitrary collection of open sets is open.}

The \emph{interior} of $E$ is the union of of all open sets that are subsets of E\footnote{Again, equivalently, this is the largest open set contained in E}. A \emph{neighborhood} of a point $p\in S$ is any open set that contains $p$. $(S, \tau)$, read $S$ equipped with the topology $\tau$, is a \emph{Hausdorff space} and $\tau$ is a \emph{Hausdorff Topology} if distinct points of $S$ have disjoint neighborhoods\footnote{Because there are real numbers between every real number, the standard topology on the reals has this property}. A set $K\subset S$ is \emph{compact} if every open cover of $K$ has a finite subcover.\footnote{Formally, if $K \subset \bigcup_{i \in \calI} U_i$ for a collection of open sets $\{U_i\}_{i\in \calI}$ then there exists a subset $J \subset \calI$ with $|J| < \infty$ such that $K \subset \bigcup_{j\in J} U_j$} 

A collection $\tau'\subset \tau$ is a \emph{base} for $\tau$ if every member of $\tau$ is a union of members of $\tau'$. A collection $\gamma$ of neighborhoods of a point $p\in S$ if every neighbothood of $p$ contains a member of $\gamma$. 

If $E\subset S$ and if $\sigma$ is the collection of all intersection $\sigma\{E \cap V: V \in \tau\}$, then we call this the topology $E$ inherits from $S$. It is easy to verify that this is a valid topology. 

If a topology $\tau$ is induced by a metric $d$, we say $d$ and $\tau$ are \emph{compatible}. 

\paragraph{Topological Vector Spaces}

Suppose $\tau$ is a topology on $X$ such that every point of $X$ is a closed set\footnote{This is the case for Hausdorff spaces} and the vector space operations are continuous with respect $\tau$. Under these condition, $\tau$ is said to be a \emph{vector topology} on $X$ and $X$ is a \emph{topological vector space}. 

A subset $E$ is said to be \emph{bounded} if, for every neighborhood $V$ of $0$ in $X$ corresponds to a number $s > 0$ such that $E \subset tV$ for every $t > s$. 

Let $X$ be a topological vector space. Associate to each $a\in X$ and each $\lambda \neq 0$ the translation operator $T_a$ and multiplication operator $M_\lambda$, by the formulas 
\[T_a(x) = a + x \andbox M_\lambda(x) = \lambda x\]
The following proposition is important 
\begin{prop}
	A homeomorphism is a continuous function with a continuous inverse. $T_a$ and $M_\lambda$ are homeomorphisms of $X$ onto $X$. 
\end{prop}
\begin{proof}
	The cector space axioms imply that $T_a$ and $M_\lambda$ are one to one, that they map $X$ onto $X$, and that their inverses are $T_{-a}$ and $M_{1/\lambda}$, respectively. We have already seen that all of these are continuous.
\end{proof}

One consequence of this is that every vector topology $\tau$ is translation invariant. That is a set $E\subset X$ is open iff each of its translates $a + E$ is open. So $\tau$ is completely determined by any local base. In the cector space contest, the term local base will always mean a local base at $0$. A local base of a topological vector space $X$ is thus a collection $\calB$ of neighborhoods of 0 such that every neighborhood of 0 contains a member of $\calB$. A metric $d$ on a vector space $X$ will be called invariant if 
\[d(x+x, y+x) = d(x,y)\]
for all $x,y,z\in X$.

\paragraph{Types of topological vector spaces}

\begin{enumerate}
	\item $X$ is locally convex if there is a local base $\calB$ whose members are convex 
	\item $X$ is locally bounded if 0 has a bounded neighborhood 
	\item $X$ is locally compact if 0 has a neighborhood whose closure is compact 
	\item $X$ is metrizable if $\tau$ is compatible with some metric $d$.
	\item $X$ is an F-space is it's topology is induced by a complete invariant metric $d$. 
	\item $X$ is a Frechet space if $X$ is a locally convex F-space.
	\item $X$ is normable if a norm exists on $X$ such that the metric induced by the norm is compatible with $\tau$.
	\item $X$ Normed spaces and Banach spaces are defined above.
	\item $X$ has the Heine-Borel property if every closed and bounded subset of $X$ is compact.
\end{enumerate}


\subsection{Seperation Properties}

\begin{theorem}
	Suppse $K$ and $C$ are subsets of a topological vector space $X$, $K$ is compact, $C$ is closed, and $K\cap C = \emptyset$  
\end{theorem}












